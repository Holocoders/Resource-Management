\chapter{Introduction}
\pagenumbering{arabic}\hspace{3mm}
The Resource management system is an application that facilitates a generalized solution for resource management in an institute or organization. The application provides a web-based portal for administrators and a mobile app for end-users.  The resources in the inventory may have an optional pipeline of booking followed by buying or renting. The wide range of applications includes booking slots in labs, borrowing lab equipment, mess management, and buying/selling of other general student commodities such as cycle, books, etc.
\section{Motivation}
Most institutes or organizations still use traditional methods such as record notebooks or excel to track and maintain their inventory. These methods are not effective and do not scale well with frequent updates. \newline
Even if there exists an automated solution, since each facility has its portal, it makes centralized resource tracking harder and also makes it hard for the end-user to keep track of each of these portals for buying or renting items. For large institutes with many facilities, maintaining each one of them separately becomes cumbersome. A unified, generalized solution is the need of the hour as it provides high scalability and agility. 
\clearpage
\section{Goals}
\begin{itemize}
    \item A generalized solution for managing various facilities in an institute or organisation.
\item A web portal for administrators to manage the inventory. 
\item A mobile application for end-users to interact with the inventory. The end-user can buy or rent items depending on their availability. 
\item The application provides secure authentication and fine-grain authorization up to the level of items for higher flexibility. 
\item The application has web-scarping features for image suggestion and a simple recommender system for items that are brought or rented together. 
\end{itemize}
\section{Organization of The Report}
This chapter provides the motivation for a resource management system and a background for the topics covered in the report. Chapter 2 explains the prior works in this domain. Chapter 3 explains the application's architecture which involves the workflow and the technical stack used in the application. Chapter 4 explains the workflow of the application. Chapter 5 explains the database design and the functionality of each entity in it. Chapter 6 shows the user interface design of the application, which includes different pages, and pop-ups. Chapter 6 has the conclusion and the future works that will be done on this application. 
